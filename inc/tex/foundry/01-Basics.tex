\newpage
% ======================================================================
\section{Basics}

% ----------------------------------------------------------------------
\subsection{Paragraphs and newlines}

Lorem ipsum dolor sit amet, consectetuer adipiscing elit. Etiam lobortis facilisis sem. Nullam nec mi et neque pharetra sollicitudin. Praesent imperdiet mi nec ante. Donec ullamcorper, felis non sodales commodo, lectus velit ultrices augue, a dignissim nibh lectus placerat pede.\par
Lorem ipsum dolor sit amet, consectetuer adipiscing elit. Etiam lobortis facilisis sem. Nullam nec mi et neque pharetra sollicitudin. Praesent imperdiet mi nec ante. Donec ullamcorper, felis non sodales commodo, lectus velit ultrices augue, a dignissim nibh lectus placerat pede.
\par
\setlength{\parindent=0pt} %No paragraph indent
This is a single line followed by a new line \\
This is a single line

% ----------------------------------------------------------------------
\subsection{Bold, Italics, Emphasize, Underline text}

This is a \textbf{bold} word. \\
This is an \textit{italicized} word. \\
This is an \emph{emphasized} word. \\
This is an \underline{underlined} word.

% ----------------------------------------------------------------------
\subsection{Unordered and ordered lists}

Ordered lists exist within an \verb|enumerate| environment:
\begin{enumerate}
  \item List entries start with the \verb|\item| command.
  \item Individual entries are indicated with bullet.
\end{enumerate}
\par

Unordered lists exist within a \verb|itemize| environment:
\begin{itemize}
  \item List entries start with the \verb|\item| command.
  \item Individual entries are indicated with bullet.
\end{itemize}

Description lists exist within a \verb|description| environment:
\begin{description}
    \item[Verb] Definition of a verb.
    \item[Indicators] Definition of an indicator.
\end{description}

% ----------------------------------------------------------------------
\subsection{Text with color}
\label{sec:textcolor}

A word can be \textcolor{red}{coloured} to any desired value.\\
A word can have a \colorbox{yellow}{colored background}.
\par
{\color{blue}
This is a paragraph of text written in blue color. Lorem ipsum dolor 
sit amet, consectetuer adipiscing elit. Lorem ipsum dolor sit amet, 
consectetuer adipiscing elit. 
}
\par
The color is reset to normal text color.

% ----------------------------------------------------------------------
\subsection{Footnote}

LaTeX, software used for typesetting \footnote{Typesetting is the way 
that text is composed using individual types — the symbols, letters and
glyphs in digital systems.} technical documents.
\par
LaTeX is a free software \footnote{the programs and other operating 
information used by a computer.} package created in 1985 by the American computer 
scientist Leslie Lamport as an addition to the TeX typesetting system. 
LaTeX was created to make it easier to produce general-purpose books and 
articles within TeX. Because LaTeX is an extension to the TeX typesetting 
system, it has TeX’s ability to typeset technical documents that contain 
complex mathematical equations.

% ----------------------------------------------------------------------
\subsection{Margin notes}

The Big Bang event is a physical theory that describes how the universe 
expanded from an initial state of high density and temperature. It was 
first proposed in 1927 by Roman Catholic priest and physicist Georges Lemaître. 
Various \marginpar{Cosmology is a phenomenal thing!} cosmological models of the Big Bang 
explain the evolution of the observable universe from the earliest known 
periods through its subsequent large-scale form. These models offer a 
comprehensive explanation for a broad range of observed phenomena, including 
the abundance of light elements, the cosmic microwave background (CMB) 
radiation, and large-scale structure. 

\par

The overall uniformity of the Universe, known as the flatness problem, is 
explained through cosmic inflation: a sudden and very rapid expansion of space 
during the earliest moments. However, physics currently lacks a widely accepted 
theory of quantum gravity that can successfully model the earliest conditions 
of the Big Bang.

% ----------------------------------------------------------------------
\subsection{Cross referencing and hyperlinks}

Section-\ref{sec:textcolor} demonstrates font and text color.
\par
Figure-\ref{fig:graph} is that of a mesh graph.
\par
For further references see \href{http://www.wikipedia.com}{Wikipedia} 
or go to the next url: \url{http://www.google.com}.
\par
\hyperlink{word:w1}{This} is a link to a word within a paragraph of text in the document.
